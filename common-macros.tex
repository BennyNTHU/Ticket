\def\ps@headings{%
\def\@oddhead{\mbox{}\scriptsize\rightmark \hfil \thepage}%
\def\@evenhead{\scriptsize\thepage \hfil \leftmark\mbox{}}%
\def\@oddfoot{}%
\def\@evenfoot{}}
%\makeatother
\pagestyle{headings}
\usepackage{epsfig, epsf, amsmath, array, amsthm, amssymb, latexsym, graphics, multirow}
%\usepackage[small]{caption}
\usepackage{graphicx}
%\usepackage{subfigure}
\usepackage{amsthm}
\usepackage{bm}
\usepackage{amsmath}
\interdisplaylinepenalty=2500

%\captionsetup{format=default,labelsep=period,justification=justified,labelfont=footnotesize,textfont=footnotesize}

\abovecaptionskip=2pt \belowcaptionskip=-12pt

\hyphenation{op-tical net-works semi-conduc-tor IEEEtran}

\newcommand {\mymarginpar}[1]{\marginpar{#1}}
\renewcommand {\marginpar}[1]{} % comment out this command to show labels in the margin

\def\_{\rule{.3em}{.15ex}}      % Get underscore by typing \_.

%%%%%%%%%%%%%%%%%%%%%%%%%%%%%%%%%%%%%%%%%%%%%%%%%%%%%%%%%%%%%%%%%%%%%%%%%%%%%%%
%%      Line Spacing (e.g., \ls{1} for single, \ls{2} for double, even \ls{1.5})
%%

%%%%%%%%%%%%%%%%%%%%%%%%%%%%%%%%%%%%%%%%%%%%%%%%%%%%%%%%%%%%%%%%%%%%%%%%%%%%
%\setlength{\textwidth}{6.25in}
%\setlength{\textheight}{9.0in}
%\setlength {\oddsidemargin}{0in}
%\setlength {\evensidemargin}{0in}
%\setlength{\topmargin}{-0.375in}

%%%%%%%%%%%%%%%%%%%%%%%%%%%%%%%%%%%%%%%%%%%%%%%%%%%%%%%%%%%%%%%%%%%%%%%%%%%%%%%
%%      Line Spacing (e.g., \ls{1} for single, \ls{2} for double, even \ls{1.5})
%%

\newcommand{\ls}[1]
   {\dimen0=\fontdimen6\the\font
    \lineskip=#1\dimen0
    \advance\lineskip.5\fontdimen5\the\font
    \advance\lineskip-\dimen0
    \lineskiplimit=.9\lineskip
    \baselineskip=\lineskip
    \advance\baselineskip\dimen0
    \normallineskip\lineskip
    \normallineskiplimit\lineskiplimit
    \normalbaselineskip\baselineskip
    \ignorespaces
   }
%%%%%%%%%%%%%%%%%%%%%%%%%%%%%%%%%%%%%%%%%%%%%%%%%%%%%%%%%%%%%%%%%%%%%%%%%%%%

% to be used in math mode:
\newcommand {\infin}{\infty}
\newcommand {\one}[1]{\mbox{$1\{ #1 \}$}}
\newcommand {\ass}{{\bf :=}}
\newcommand {\caret}{\widehat{~~}}
\newcommand {\bearn}{\begin{eqnarray*}}
\newcommand {\eearn}{\end{eqnarray*}}
\newcommand {\barr}{\begin{array}}
\newcommand {\earr}{\end{array}}
\newcommand {\cee}[2] {\left( \begin{array}{c}
         #1 \\ #2 \end{array} \right)}
\newcommand {\Set}{{\cal S}}
\newcommand {\A}{{\cal A}}
\newcommand {\B}{{\cal B}}
\newcommand {\C}{{\cal C}}
\newcommand {\D}{{\cal D}}
\renewcommand {\L}{{\cal L}}
\newcommand {\RR}{{\cal R}}
\newcommand {\U}{{\cal U}}
\newcommand {\Un}{{\cal U}^n}
\newcommand {\V}{{\cal V}}
\newcommand {\W}{{\cal W}}
\newcommand {\Um}{{\cal U}^m}
\newcommand {\Z}{{\cal Z}}
\newcommand {\F}{{\bf F}}
\newcommand {\Nc}{{\cal N}_c}
\newcommand {\N}{{\cal N}}
\newcommand {\Sc}{{\cal S}_c}
\renewcommand {\S}{{\cal S}}
\newcommand {\Vc}{{\cal V}_c}
\newcommand {\Y}{{\cal Y}}
\newcommand {\X}{{\cal X}}
\newcommand {\xp}{x^\prime}
\newcommand\cf{{\cal F}}
\newcommand\cg{{\cal G}}
\newcommand\app{\approx}
\newcommand{\calP}{\mathcal{P}}
\newcommand{\calS}{\mathcal{S}}

%*****************PROBABILITY*****************
\newcommand {\given}{\; | \;}
\newcommand \absolute[1]{\left | #1 \right |}
%\renewcommand {\[}{[ \:}
%\renewcommand {\]}{\: ] \,}

%******************** CONvergence of  rv's
\newcommand\almostsure{\stackrel{a.s.}{\longrightarrow} }
\newcommand\indist{\stackrel{D}{\longrightarrow} }
\newcommand\inprob{\stackrel{P}{\longrightarrow} }

%\newcommand\prob[1]{P \left [ \: #1 \; \right ] }
%\newcommand\probg[2]{P \[ #1 \given #2 \] }
\newcommand\expect[1]{E \left [ \: #1 \: \right ] }
\newcommand\var[1]{\sigma_{#1}}
\newcommand\Var[1]{Var \left ({#1} \right )}
\newcommand\vartwo[1]{\sigma_{#1}^2}
\newcommand\expecttwo[2]{E_{#2} \left [ \: #1 \: \right ] }
\newcommand\xbar[1]{{\overline #1}}
\newcommand\xbartwo[2]{\left ({\overline #1} \right )      ^{#2}}
\newcommand\expectg[2]{E \[ #1 \given #2 \] }
\newcommand\indicator[2]{I_{#1}(#2)}
\newcommand\indicate[1]{I\left \{{#1}\right \}}
\newcommand\generate[1]{{ \cal G}_{#1}}
\newcommand\moment[1]{{ \cal M}_{#1}}
\newcommand\laplace[1]{{\cal L}_{#1}}

\newcommand\equalst{=_{st}}
\newcommand\greaterst{>_{st}}
\newcommand\lesserst{<_{st}}
\newcommand\suchthat{\,:\,}
\newcommand\impulse[1]{u_0(#1)}

\newcommand\hospital{L'Hospital's rule }
\newcommand\Lr{L'Hospital's rule }
\newcommand\Rt{Rouch\'{e}'s theorem }
\def\defeq{\stackrel{\scriptstyle\rm def}{=}}
\def\twoLineSub#1#2{{#1}\atop{#2}}

%******************  NEWTHEOREMS**********************
\newtheorem{definition}{Definition}
\newtheorem{property}[definition]{Property}
\newtheorem{proposition}[definition]{Proposition}
\newtheorem{lemma}[definition]{Lemma}
\newtheorem{theorem}[definition]{Theorem}
\newtheorem{corollary}[definition]{Corollary}
\newtheorem{example}[definition]{Example}
\newtheorem{remark}[definition]{Remark}
%\newtheorem{axiom}[definition]{Axiom}
\newtheorem{algorithm}[definition]{Algorithm}
\newtheorem{assumption}[definition]{Assumption}



%\newtheorem{definition}{Definition}
%\newtheorem{property}{Property}
%\newtheorem{lemma}{Lemma}
%\newtheorem{theorem}{Theorem}
%\newtheorem{corollary}{Corollary}


%********************Random V ariables****************************

\newcommand{\degenerate}[1]
{\mbox {\bf  $\tilde D_{#1}$}}

\newcommand{\binomial}[3]
{\mbox {\bf  $\tilde B_{#1,#2}^{#3}$}}

\newcommand{\geometric}[3]
{\mbox {\bf  $\tilde G_{#1,#2}^{#3}$}}

\newcommand{\poisson}[2]
{\mbox {\bf  $\tilde P_{#1}^{#2}$}}

\newcommand{\erlang}[3]
{\mbox {\bf  $\tilde E_{#1,#2}^{#3}$}}

\newcommand{\uniform}[2]
{\mbox {\bf  $\tilde U_{#1}^{#2}$}}

\newcommand{\normal}[3]
{\mbox {\bf  $\tilde N_{#1,#2}^{#3}$}}



%********************** FACTORIAL EXPRESSIONS*****************
\newcommand{\comb}[2]
{\left ( \begin{array}{c} #1 \\#2 \end{array} \right ) }

\newcommand{\inlinecomb}[2]
{\mbox { \scriptsize $ \left ( \begin{array}{c} #1 \\#2
\end{array} \right ) $ \normalsize}}

\newcommand{\combr}[2] {
\left \langle \begin{array}{c} #1 \\#2
\end{array} \right \rangle }

\newcommand{\inlinecombr}[2]
{\mbox{ \scriptsize $ \left \langle \begin{array}{c} #1 \\#2
\end{array} \right \rangle
$ \normalsize}}

\newcommand{\stirltwo}[2] {
\left \{ \begin{array}{c} #1 \\#2
\end{array} \right \} }

\newcommand{\inlinestirltwo}[2]
{\mbox{ \scriptsize $ \left \{ \begin{array}{c} #1 \\#2
\end{array} \right \}
$ \normalsize}}


\newcommand{\stirlone}[2] {
\left [ \begin{array}{c} #1 \\#2
\end{array} \right ] }

\newcommand{\inlinestirlone}[2]
{\mbox{ \scriptsize $ \left [ \begin{array}{c} #1 \\#2
\end{array} \right ]
$ \normalsize}}

%\newcommand{\fact}[2]{(#1)_{#2}}
\newcommand{\factr}[2]{\langle #1\rangle _{#2}}

%*********************************LISTS*****************************
\newcommand {\benum} {\begin{enumerate}}
\newcommand {\eenum} {\end{enumerate}}

\newcommand {\bdesc} {\begin{description}}
\newcommand {\edesc} {\end{description}}
\newcommand {\ix}[1] {\index{#1}}

%************************INTEGRALS*******************************
\newcommand {\intlimits}[3] {\left. #1 \right |_{#3}^{#2}}

%************************** FIGURES*******************************
% New version
\newcommand {\bfig}[2] {\begin{figure}[htbp]
                        \centerline {
                         \epsfig{figure={#1},clip=,width={#2}}}}
\newcommand {\brotatefig}[2] {\begin{figure}[htbp]
                        \centerline {
                         \epsfig{figure={#1},clip=,angle=-90,width={#2}}}}
% Old version
%\newcommand {\bfig}[2] {\begin{figure}[p]
%                            \centerline {
%                            \setlength{\epsfxsize}{#2}
%                            \epsffile{#1}}}

% Skip version
%\newcommand {\bfig}[2] {\begin{figure}[p]}
%\newcommand {\brotatefig}[2] {\begin{figure}[p]}


\newcommand {\bfigfirst}[2] {\begin{figure}[h]
                        \centerline {
                        \setlength{\epsfxsize}{#2}
                        \epsffile{#1}}}
\newcommand {\efig}[2]{ \caption{#2}
                        \label{fig:#1}
                        \end{figure}
                        \centerline {
                        \mymarginpar{fig:#1}}}
\newcommand {\erotatefig}[2]{ \caption{#2}
                        \label{fig:#1}
                        \end{figure}
                        \mymarginpar{fig:#1}}
\newcommand {\rfig}[1]{Figure \ref{fig:#1}}

%************************** TABLES********************************
\newcommand {\btab}[1]{
                       \begin{table}
                       \centering
                       \begin{tabular}{#1}}
\newcommand {\etab}[3] {
                       \end{tabular}
                       \caption[#3]{#2}
                       \label{tab:#1}
                       \end{table}
                       \mymarginpar{tab:#1}
                       \vspace{.1in}}
\newcommand {\rtab}[1]{Table \ref{tab:#1}}

\newcommand {\btabular}[1]{\begin{center}
                       \begin{tabular}{#1}}
\newcommand {\etabular}{\end{tabular}
                       \end{center}}

%************************** DEFINITIONS********************************
\newcommand {\bdefin}[1]{\begin{definition}
                      \mymarginpar{def:#1}
                      \label{def:#1} }
\newcommand {\edefin}       {\end{definition}}
\newcommand {\rdef}[1]{Definition \ref{def:#1}}
%************************** ASSUMPTIONS********************************
\newcommand {\bassum}[1]{\begin{assumption}
                      \mymarginpar{ass:#1}
                      \label{ass:#1} }
\newcommand {\eassum}       {\end{assumption}}
\newcommand {\rass}[1]{Assumption \ref{ass:#1}}

%************************** PROPERTY********************************
\newcommand {\bpro}[1]{\begin{property}
                      \mymarginpar{pro:#1}
                      \label{pro:#1} }
\newcommand {\epro}   {\end{property}}
\newcommand {\rpro}[1]{Property \ref{pro:#1}}

%************************** PROPOSITION********************************
\newcommand {\bprop}[1]{\begin{proposition}
                      \mymarginpar{prop:#1}
                      \label{prop:#1} }
\newcommand {\eprop}       {\end{proposition}}
\newcommand {\rprop}[1]{Proposition \ref{prop:#1}}

%************************** LEMMA********************************
\newcommand {\blem}[1]{\begin{lemma}
                      \mymarginpar{lem:#1}
                      \label{lem:#1} }
\newcommand {\elem}   {\end{lemma}}
\newcommand {\rlem}[1]{Lemma \ref{lem:#1}}

%************************** THEOREM******************************
\newcommand {\bthe}[1]{\begin{theorem}
                      \mymarginpar{the:#1}
                      \label{the:#1} }
\newcommand {\ethe}   {\end{theorem}}
\newcommand {\rthe}[1]{Theorem \ref{the:#1}}

%************************** PROOF******************************
\newcommand {\bproof}[1]{\noindent {\bf Proof #1.} \ }
\newcommand {\eproof} {\hfill \squares \\ \vspace{.3cm}}
%************************** COROLLARY******************************
\newcommand {\bcor}[1]{\begin{corollary}
                      \mymarginpar{cor:#1}
                      \label{cor:#1} }
\newcommand {\ecor}   {\end{corollary}}
\newcommand {\rcor}[1]{Corollary \ref{cor:#1}}

%************************** AXIOMS******************************
\newcommand {\bax}[1]{\begin{axiom}
                      \mymarginpar{ax:#1}
                      \label{ax:#1} }
\newcommand {\eax}       {\vspace{-.1in} \end{axiom}}
\newcommand {\rax}[1]{Axiom \ref{ax:#1}}

%************************** EXAMPLES **********************************
%\newcommand {\bex}[2]{\vspace{.2in}
%                          \begin{example}
%                          \mymarginpar{ex:#1}
%                          \small
%                          {\bf #2 }
%                          \label{ex:#1} }
%\newcommand {\eex}   {\end{example} \vspace{.3cm} \normalsize}
\newcommand {\bex}[2]{\vspace{.1in}
                      \begin{example}
                      \mymarginpar{ex:#1}
                       {\bf #2}
                      \label{ex:#1} \em}
\newcommand {\eex}       {\end{example} \vspace{.3cm} }
\newcommand {\rex}[1]{Example \ref{ex:#1}}

%************************** REMARK******************************
\newcommand {\brem}[1]{\begin{remark}
                      \mymarginpar{rem:#1}
                      \label{rem:#1} \em }
\newcommand {\erem}   {\end{remark}}
\newcommand {\rrem}[1]{Remark \ref{rem:#1}}

%************************** EQUATIONS**********************************
\newcommand {\beq}[1]{\mymarginpar{eq:#1}
                      \begin{equation}
                      \label{eq:#1} }

\newcommand {\beqno}[1]{\mymarginpar{eq:#1}
                      \begin{eqnarray}
                      \nonumber}

\newcommand {\eeq}       {\end{equation}}
\newcommand {\eeqno}       { && \end{eqnarray}}
\newcommand {\req}[1]{(\ref{eq:#1})}
\newcommand {\rear}[1]{(\ref{eqar:#1})}

\newcommand {\bear}[1]{\mymarginpar{eq:#1}
                       \begin{eqnarray}
                       \label{eq:#1} }

\newcommand {\bearno}[1]{\mymarginpar{eq:#1}
                       \begin{eqnarray}
                       \nonumber}

\newcommand {\eear}{\end{eqnarray}}
\newcommand {\eearno}{\end{eqnarray}}
%*****************SELECTION IN MATH*****************************
\newcommand {\bsel}{\left \{ \begin{array}{cl}}
\newcommand {\esel}{\end{array} \right.}

%*****************MATRICES IN MATH*****************************
\newcommand {\bmat}[1]{\left [ \begin{array}{#1}}
\newcommand {\emat}{\end{array} \right ]}
%************************** SECTIONS**********************************
\newcommand {\bsec}[2]{\mymarginpar{sec:#2}
                       \section{#1}
                       \label{sec:#2} }

\newcommand {\rsec}[1]{Section \ref{sec:#1}}

%***************************CHAPTER************************************
\newcommand {\rcha}[1]{Chapter \ref{cha:#1}}

%************************** SUBSECTIONS**********************************
\newcommand {\bsubsec}[2]{\mymarginpar{sec:#2}
                       \subsection{#1}
                       \label{sec:#2} }

\newcommand {\rsubsec}[1]{Section \ref{sec:#1}}

\newcommand {\heading}[1]{\vspace{.4in}
                          \noindent
                          \addcontentsline{toc}{subsection}
                          {\hspace{.5in} {\em #1}}
                           {\bf #1}
                           \vspace{.15in}}

\newcommand {\headingtwo}[1]{\vspace{.4in}
                          \noindent
                          \addcontentsline{toc}{subsection}
                          {\hspace{1in} {\em #1}}
                           {\bf #1}
                           \vspace{.15in}}
%************************** SUBSUBSECTIONS**********************************
\newcommand {\bsubsubsec}[2]{\mymarginpar{sec:#2}
                       \subsubsection{#1}
                       \label{sec:#2} }

\newcommand {\rsubsubsec}[1]{Section \ref{sec:#1}}

\newcommand {\subheading}[1]{\vspace{.4in}
                          \noindent
                          \addcontentsline{toc}{subsection}
                          {\hspace{.5in} {\em #1}}
                           {\bf #1}
                           \vspace{.15in}}

\newcommand {\subheadingtwo}[1]{\vspace{.4in}
                          \noindent
                          \addcontentsline{toc}{subsection}
                          {\hspace{1in} {\em #1}}
                           {\bf #1}
                           \vspace{.15in}}
%************************** PROBLEMS****************************


\def\R{I\kern-0.30em R}
\def\N{I\kern-0.30em N}
\def\P{I\kern-0.30em P}

%*********** COmmands for Continuous Probability*******************
%\def\bibitem{\@ifnextchar[{\@lbibitem}{\@bibitem}}


%\def\sqr#1#2{{\vcenter{\hrule height.#2pt
%         \hbox{\vrule width.#2pt height#1pt \kern#1pt
%           \vrule width.#2pt}
%      \hrule height.2pt}}}
% \def\square{$\sqr79$}
\newcommand\squares{\vrule height6pt width7pt depth1pt}



\newcommand{\de}{\buildrel \rm def \over =}
\renewcommand{\de}{\equiv}




% Macros used in this paper only


% Macros used in this paper only

\newcommand{\pib}{p}
\newcommand{\pic}{{\tilde p}}
\def\ex{{\bf\sf E}}
\def\pr{{\bf\sf P}}
\newcommand{\stackunder}[2]{\mathop{#2}\limits_{#1}}
%\newcommand{\text}[1]{\mbox{#1}}

\newcommand{\qz}{q^0}
\newcommand{\ain}{a}
\newcommand{\bin}{b}

\def\bfa{{\mbox{$\bm{a}$}}}
\def\bfb{{\mbox{$\bm{b}$}}}
\def\bfe{{\mbox{$\bm{e}$}}}
\def\bfd{{\mbox{$\bm{d}$}}}
\def\bfsd{{\mbox{\scriptsize$\bm{d}$}}}
\def\bff{{\mbox{$\bm{f}$}}}
\def\bfsf{{\mbox{\scriptsize$\bm{f}$}}}
\def\bfg{{\mbox{$\bm{g}$}}}
\def\bfh{{\mbox{$\bm{h}$}}}
\def\bfl{{\mbox{$\bm{\ell}$}}}
\def\bfm{{\mbox{$\bm{m}$}}}
\def\bfn{{\mbox{$\bm{n}$}}}
\def\bfsn{{\mbox{\scriptsize$\bm{n}$}}}
\def\bfp{{\mbox{$\bm{p}$}}}
\def\bfq{{\mbox{$\bm{q}$}}}
\def\bfr{{\mbox{$\bm{r}$}}}
\def\bfsr{{\mbox{\scriptsize$\bm{r}$}}}
\def\bfs{{\mbox{$\bm{s}$}}}
\def\bft{{\mbox{$\bm{t}$}}}
\def\bfu{{\mbox{$\bm{u}$}}}
\def\bfv{{\mbox{$\bm{v}$}}}
\def\bfx{{\mbox{$\bm{x}$}}}
\def\bfy{{\mbox{$\bm{y}$}}}
\def\bfw{{\mbox{$\bm{w}$}}}
\def\bfsx{{\mbox{\scriptsize$\bm{x}$}}}
\def\bfN{{\mbox{$\bm{N}$}}}
\def\bfA{{\mbox{$\bm{A}$}}}
\def\bfB{{\mbox{$\bm{B}$}}}
\def\bfC{{\mbox{$\bm{C}$}}}
\def\bfD{{\mbox{$\bm{D}$}}}
\def\bfE{{\mbox{$\bm{E}$}}}
\def\bfF{{\mbox{$\bm{F}$}}}
\def\bfG{{\mbox{$\bm{G}$}}}
\def\bfI{{\mbox{$\bm{I}$}}}
\def\bfJ{{\mbox{$\bm{J}$}}}
\def\bfsN{{\mbox{\scriptsize$\bm{N}$}}}
\def\bfM{{\mbox{$\bm{M}$}}}
\def\bfP{{\mbox{$\bm{P}$}}}
\def\bfQ{{\mbox{$\bm{Q}$}}}
\def\bfU{{\mbox{$\bm{U}$}}}
\def\bfV{{\mbox{$\bm{V}$}}}
\def\bfX{{\mbox{$\bm{X}$}}}
\def\bfY{{\mbox{$\bm{Y}$}}}
\def\bfZ{{\mbox{$\bm{Z}$}}}
\def\bfpi{{\mbox{$\bm{\pi}$}}}
\def\bfxi{{\mbox{$\bm{\xi}$}}}
\def\bfone{{\mbox{$\bm{1}$}}}
\def\bfzero{{\mbox{$\bm{0}$}}}

\def\bfj{{\mbox{$\bm{j}$}}}
\def\bfi{{\mbox{$\bm{i}$}}}
\def\bfT{{\mbox{$\bm{T}$}}}

\def\grad{{\nabla}}
%\def\stge{\stackrel{\ge}{\mbox{st}}}
\def\stge{\ge_{\rm st}}
\def\twoLineSub#1#2{{#1}\atop{#2}}
\def\argmin{\mathop{\rm argmin}}
\def\argmax{\mathop{\rm argmax}}
\def\indicatorFunction#1{{1_{\left\{#1\right\}}}}

\newcommand{\erdos}{Erd\H{o}s\ }
\newcommand{\renyi}{R\'{e}nyi\ }
\newcommand{\er}{Erd\H{o}s and R\'{e}nyi\ }
\newcommand{\edr}{Erd\H{o}s-R\'{e}nyi\ }
\newcommand{\binomcoeff}[2]{\left(\begin{array}{c}#1\\#2\end{array}\right)}
\newcommand{\noeqarwidth}{\setlength{\arraycolsep}{0pt}}




\newcommand{\gc}{\emph{greedy construction}}
\newcommand{\tba}{\centerline{\em *** To Be Added ***}}

%\bibliographystyle{IEEE}


%\newtheorem{part}{Part}
